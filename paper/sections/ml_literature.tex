% ML Literature for DDEs

Machine learning with delays appears in several ways:
\begin{enumerate}
  \item[(i)] continuous-depth models explicitly parameterized as neural DDEs,
  \item[(ii)] learning unknown delays as part of system identification,
  \item[(iii)] PINN-style solvers adapted to delayed residuals,
  \item[(iv)] operator-learning methods (e.g., DeepONets, neural operators) trained on paired data.
\end{enumerate}

\subsection{Neural Delay Differential Equations}

Neural DDEs extend Neural ODEs by introducing explicit delayed arguments and adjoint-based training.
The key idea is to parameterize the right-hand side of a DDE with a neural network:
\begin{equation}
  \dot{x}(t) = f_\theta\bigl(t, x(t), x(t-\tau)\bigr),
\end{equation}
where $f_\theta$ is a neural network. Training proceeds via the adjoint method, extended to
handle the delayed terms~\citep{zhu2021ndde}.

\subsection{Learning Unknown Delays}

Recent work addresses the problem of learning the delay parameters themselves from data.
\citet{stephany2024learningdelay} propose methods for inferring unknown delays in DDEs from
observed trajectories, which is important for system identification in applications where
the delay structure is not known a priori.

\subsection{State and Time-Dependent Neural DDEs}

Extensions to state-dependent and time-dependent delays have been explored~\citep{monsel2024sddde},
which significantly increases the expressiveness of neural DDE models but also introduces
additional computational challenges.

\subsection{Physics-Informed Neural Networks for DDEs}

Physics-informed approaches have been adapted for DDEs by incorporating the delayed residual
into the loss function. However, handling the history function and the non-local nature of
the delay term requires careful treatment.

\subsection{Operator Learning}

Operator learning methods such as DeepONets~\citep{wang2021pideeponet} and Fourier Neural
Operators~\citep{li2020fno} learn mappings between function spaces. For DDEs, the natural
formulation is to learn the operator from history functions (and parameters) to solution
trajectories, which is the approach we pursue in this work.
